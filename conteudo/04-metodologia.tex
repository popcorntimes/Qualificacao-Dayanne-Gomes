%!TeX root=../tese.tex
%("dica" para o editor de texto: este arquivo é parte de um documento maior)
% para saber mais: https://tex.stackexchange.com/q/78101/183146

\chapter{Metodologia}
\label{chap:metodologia}

Neste capítulo, definimos a metodolofia empregada para responder as questões levantada na pesquisa. As bases de dados utilizadas estão descritas na \autoref{sec:bases}. Já as etapas do processo de ciência de dados, que explora os dados obtidos, estão descritas na seções subsequentes. O pré-processamento está descrito na \autoref{sec:pre-processamento} e a análise exploratória de dados está descrita na \autoref{sec:eda}.

\section{Bases de dados}
\label{sec:bases}
Nesta seção, descrevemos as bases de dados utilizadas nesta etapa da pesquisa. Para realizar a análise das escolhas de graduação, utilizamos dados educacionais dos estudantes. Uma das fontes é o Exame Nacional do Ensino Médio (ENEM). Através dele, conseguimos obter informações sobre alunos concludentes e que já concluíram o ensino médio, bem como de suas escolas. A outra fonte utilizada é o Sistema de Seleção Unificada (SISU). Com ela, obtemos informações relativas à inscrição dos alunos em cursos de nível superior, além de detalhes das instituições e cursos ofertados. Uma versão resumida dos dicionários de dados, que explicam as variáveis das bases, está disponível no Apêndice A. 

\subsection{ENEM}
O Exame Nacional do Ensino Médio (ENEM) é um exame realizado pelo Ministério da Educação cujo objetivo é avaliar o desempenho escolar no final da educação básica. Desde 2009, ele passou a ser utilizado como mecanismo de ingresso à educação superior, cujas notas podem ser aproveitadas no Sistema de Seleção Unificada (SISU) e Programa Universidade para Todos (ProUni). O exame também possibilita o pleito de certificação do ensino médio. Os participantes realizam provas em quatro áreas de conhecimento: linguagens, ciências humanas, ciências da natureza e matemática. Além disso, eles devem desenvolver um texto dissertativo-argumentativo dada uma situação-problema, conhecido como redação \autocite{inep:1}.

Os dados do Exame Nacional do Ensino Médio são disponibilizados publicamente através do Instituto Nacional de Estudos e Pesquisas Educacionais Anísio Teixeira (INEP). Os dados abertos do INEP incluem os microdados do ENEM, que reúnem um conjunto de informações relativas ao exame. Os microdados são o menor nível de desagregação de dados recolhidos, sendo disponibilizados dados dos anos de 1998 a 2022. Nesta etapa preliminar da pesquisa, utilizamos os dados do ano de 2016. Além dos microdados relativos às edições anuais, também são disponibilizados outros arquivos relevantes, como dicionário de dados, documentos técnicos, provas, gabaritos e programa para leitura da base.

Com o passar dos anos, os microdados foram se diferenciando à medida que eram incluídas ou retiradas determinadas variáveis, mas pode-se observar uma estrutura comum entre as edições. Em 2016, os dados estão divididos nas categorias:

\begin{itemize}
  \item Dados do participante; 
  \item Dados da escola;
  \item Dados dos pedidos de atendimento especializado;
  \item Dados dos pedidos de atendimento específico;
  \item Dados dos pedidos de recursos especializados e específicos para realização das provas;
  \item Dados dos pedidos de certificação do ensino médio;
  \item Dados do local de aplicação da prova;
  \item Dados da prova objetiva;
  \item Dados da redação;
  \item Dados do questionário socioeconômico.
\end{itemize}

Como parte da política adotada pela Lei Geral de Proteção de Dados Pessoais (LGPD), os dados passam por um tratamento antes de serem publicados. Isso significa que dados cadastrais e sensíveis, como nome, endereço, RG, etc, não são disponibilizados ou passam por uma máscara para anonimizá-lo, como é o caso do número de inscrição. Os arquivos de microdados são disponibilizados no formato .csv (valores separados por vírgulas). Os dados de 2016 constituem-se por apenas um arquivo, com uma tabela. Cada linha da tabela representa a inscrição de um candidato de forma individual, bem como as colunas são as variáveis definidas anteriormente, que caracterizam o participante. 

\subsection{SISU}
O Sistema de Seleção Unificada (SISU) é um sistema eletrônico do Ministério da Educação, no qual instituições públicas de ensino superior de todo o Brasil oferecem vagas para estudantes participantes do Exame Nacional do Ensino Médio. Durante o período da oferta de vagas, os alunos são ranqueados de acordo com as notas no exame e, aqueles com melhor classificação, são selecionados. Em cada processo seletivo do SISU, que tem duas aberturas anuais, o candidato pode escolher até duas opções de curso. É possível verificar informações sobre as vagas oferecidas, como cursos, instituições e localizações, turnos e modalidade de concorrência \autocite{mec:1}.

No ato da inscrição, o sistema recupera as notas da edição mais recente anterior do ENEM. Por exemplo, o SISU 2023 leva em consideração a edição do ENEM 2022. Apenas aqueles que obtiveram nota superior a zero na redação e não têm o status de treineiro no ENEM podem se inscrever. O processo é totalmente digital e gratuito, sendo o estudante o responsável por acompanhar o status da sua inscrição durante o mesmo. Quando não há a aprovação em uma das duas opções selecionadas, conhecido por chamada regular, ainda é possível a disputa por vaga através da lista de espera.

Os dados do Sistema de Seleção Unificada foram obtidos através do Portal de Dados Abertos do MEC. O portal é uma plataforma que disponibiliza dados e informações públicas do Ministério da Educação, que podem ser usadas no desenvolvimento de aplicativos e ações. Além do SISU, é possível observar conjuntos de dados de outros programas como FIES, ProUni e PRONATEC. São disponibilizados dados relacionados às inscrições realizadas nos processos seletivos dos anos de 2017 a 2022.  Nesta etapa preliminar da pesquisa, utilizamos os dados do ano de 2017.

São fornecidas informações detalhadas sobre o participante, como dados pessoais e desempenho nas provas do ENEM, a vaga para qual ele se inscreve, além da classificação e aprovação. Diferente do ENEM, não há especificação de categoria dos dados no dicionário fornecido. Também há a aplicação de máscara para anonimizar dados sensíveis, como CPF e número de inscrição. Os arquivos também são disponibilizados no formato .csv. Desta vez, são constituídos por múltiplas tabelas. Cada tabela representa uma etapa do processo de convocação dos candidatos, sendo divididas entre chamada regular e lista de espera. Ocorre duas chamadas regulares ao longo do ano, uma em cada semestre. Já a quantidade de listas de espera varia, conforme o preenchimento de vagas nas etapas anteriores. Nesta etapa da pesquisa, utilizamos apenas os dados das chamadas regulares. Cada linha da tabela representa uma inscrição de um candidato, sendo possível que um candidato tenha múltiplas inscrições por conta das duas aberturas do processo ao longo do ano e por poder se inscrever em mais de um curso.

\subsection{Censo Escolar}
O Censo Escolar é um levantamento de informações da educação básica brasileira em escolas e instituições de ensino por todo o país. Essa ferramenta demográfica realiza coletas anuais em colaboração entre o Inep e as secretarias estaduais e municipais de educação, contando com a participação de todas as escolas públicas (federais, estaduais e municipais) e privadas da rede de ensino. O Censo abrange diferentes etapas e modalidades de ensino da educação básica e profissional. Ele permite a obtenção de dados individualizados, em diversos aspectos, de estudantes, professores, turmas e escolas. A pesquisa é realizada em duas etapas: a primeira coleta informações sobre os estabelecimentos de ensino, gestores, turmas, alunos e profissionais escolares em sala de aula; a segunda, informações sobre o movimento e o rendimento escolar dos alunos. Os dados do Censo Escolar, de forma semelhante aos anteriores, são disponibilizados ao público pelo INEP no formato .csv. 


\section{Pré-processamento}
\label{sec:pre-processamento}
Na etapa da metodologia de pré-processamento, os dados são tratados a fim de adaptá-los às necessidades do projeto.
Com isso, otimizamos as etapas posteriores através de obtenção de um conjunto de dados que seja mais relevante para a pesquisa, facilitando o processo de análise. 
As técnicas aplicadas podem ser agrupadas em redução, integração, limpeza e transformação de dados. Utilizamos como referência os trabalhos de \citet{Jafari2022, Garcia2016},
que provêm definições e exemplos práticos de como realizar esses processos.

É importante frisar que o processo de ciência de dados não é rígido e estático, mas sim um processo que se flexibiliza à medida que novas necessidades vão surgindo durante o projeto. Assim, as técnicas aplicadas no pré-processamento também são utilizadas além da etapa inicial. Para isso, utilizamos a linguagem Python, com as bibliotecas pandas e NumPy para análise e manipulação de dados.

Os datasets originais possuem um grande volume de dados, tanto pela quantidade de participantes inscritos, 
quanto pela quantidade de informações armazenadas sobre eles, expressas pelas colunas (ou \textit{features}). 
Uma consequência disso é uma maior utilização de recursos computacionais para processá-los, seja nos processos 
de leitura e escrita de arquivo quanto no armazenamento. Isso é particularmente importante pela inclusão de múltiplas fontes de dados. 
Assim, visamos diminuir a quantidade de dados pouco relevantes para a pesquisa, gerando dados menos volumosos e mais representativos. 

Uma técnica aplicada na redução foi a seleção de \textit{features}. Na seleção de \textit{features}, visando diminuir a dimensionalidade
 do conjunto de dados, é gerado um subconjunto das \textit{features} originais através da identificação e remoção daquelas pouco relevantes
ou redundantes. Isso foi realizado no \textit{dataset} do ENEM, que possui uma grande quantidade de colunas, e algumas informações, 
como as referentes à aplicação da prova, não são importantes na análise a ser realizada. 
Outra técnica aplicada foi a tipagem explícita de dados. Uma particularidade da biblioteca utilizada, pandas, é que a inferência de tipos 
nem sempre é a mais eficiente, o que pode levar a uma utilização de memória maior que o esperada, além de dificultar operações específicas, 
como manipulação de \textit{strings} e realização de cálculos matemáticos. Para contornar esse problema, fizemos uma análise dos valores, a fim de mapeá-los
para tipos de dados mais precisos, que pudessem melhorar os resultados da análise.

Outra característica dos \textit{datasets} é que eles são imperfeitos. Apesar da presença de documentação auxiliar que define como os dados se comportam, na prática, os dados têm um estado diferente, incompletos e com "sujeiras". Como a qualidade dos dados interfere nos resultados obtidos, foi necessário observar características dos dados para definir uma abordagem adequada.
Um dos problemas notados foi a ausência de valores em alguns campos. Por exemplo, foram encontrados registros de inscrições do SISU que não possuíam o curso de graduação do candidato. Optamos por não descartar esses registros que tivessem valores faltando, já que poderia causar uma perda de acurácia e um viés de auto-seleção, bem como a desconsideração de algumas \textit{features} mais importantes que outras. 

Visando solucionar essa questão, criamos novas classes de valores para representar uma instância de informação ausente e, quando possível, utilizamos funções de probabilidade para inserir valores inferidos. Também tornamos os dados consistentes entre si, já que para a combinação das bases de dados, é necessário que os valores sejam do mesmo tipo e estejam uniformes em cada uma elas. Ao realizarmos uma análise de anomalias para detectar valores problemáticos, observamos uma situação particular na qualidade dos dados do SISU. Por conter mais de um \textit{dataset} da chamada regular, era necessário agrupá-los para obter o cenário do ano como um todo, mas um deles estava parcialmente corrompido. A utilização inadvertida de dados não-estruturados pode levar a interpretações incorretas e falsas conclusões no processo de análise. Assim, realizamos um filtro dessas anomalias, relacionadas aos nomes das instituições de ensino, seus campi e cursos de graduação, que foram identificadas e tratadas para refletir o comportamento esperado. 

\section{Análise exploratória de dados}
\label{sec:eda}
A partir da obtenção dos conjuntos de dados pré-processados, pudemos realizar uma análise exploratória de dados. Nesta etapa, iremos utilizar técnicas de manipulação de dados e ferramentas estatísticas para investigar os dados. Nela, conseguimos ter um entendimento melhor do cenário geral a ser explorado, como os dados estão distribuídos, quais são as variáveis, como elas se relacionam entre si e como podemos utilizá-las para responder as questões de pesquisa.

Por definição, o \textit{dataset} do ENEM possui os dados do exame de estudantes de todo o país, que totalizam 8627367 registros. Cada registro corresponde a inscrição de um único estudante. No ato de realização do exame, esses participantes estão em diferentes situações em relação ao ensino médio, podendo já ter concluído, estar cursando ou não ter concluído e não estar cursando. Essa informação está expressa em duas variáveis: TP\_ST\_CONCLUSAO, Situação de conclusão do Ensino Médio, e Q046, que é a resposta do questionário socioeconômico à pergunta "Você já concluiu ou está concluindo o Ensino Médio?", visualizadas na \autoref{tab:situacao-em}. Em ambas, mais da metade dos estudantes já concluíu o ensino médio, seguidos daqueles que estão cursando e irão concluir em 2016.

\begin{table}[h]
  \begin{tabular}{lcccc}
  \hline
  \multicolumn{1}{c}{\multirow{2}{*}{\textbf{Situação}}}                                         & \multicolumn{2}{c}{\textbf{TP\_ST\_CONCLUSAO}} & \multicolumn{2}{c}{\textbf{Q046}}         \\ \cline{2-5} 
  \multicolumn{1}{c}{}                                                                           & \textbf{Quantidade}    & \textbf{Percentual}   & \textbf{Quantidade} & \textbf{Percentual} \\ \hline
  Já concluí o Ensino Médio                                                                      & 4928251                & 57,12\%               & 4947935             & 57,35\%             \\ \hline
  \begin{tabular}[c]{@{}l@{}}Estou cursando e concluirei\\ o Ensino Médio em 2016\end{tabular}   & 1882278                & 21,82\%               & 1872570             & 21,70\%             \\ \hline
  \begin{tabular}[c]{@{}l@{}}Estou cursando e concluirei\\ o Ensino Médio após 2016\end{tabular} & 1344085                & 15,58\%               & 1331073             & 15,43\%             \\ \hline
  \begin{tabular}[c]{@{}l@{}}Não concluí e não estou\\ cursando o Ensino Médio\end{tabular}      & 472753                 & 5,48\%                & 475785              & 5,51\%              \\ \hline
  \end{tabular}
  \caption{Situação de conclusão do ensino médio dos candidatos do ENEM}
  \label{tab:situacao-em}
  \end{table}

  Outros dados relevantes são o ano e o tipo de ensino em que o participante concluíu o ensino médio. A primeira informação está expressa na variável TP\_ANO\_CONCLUIU, que explicita dos anos anteriores ao exame até 2007, agrupando o restante em anterior a 2007 ou não informado, que pode ser visualizado na \autoref{tab:ano-conclusao}. No caso de anos não informados, isso acontece tanto por haver valores ausentes da fonte, quanto pelos alunos concluíntes não terem um ano de conclusão especificado. A segunda está expressa em TP\_ENSINO, que descreve o tipo de ensino da instituição na qual o aluno concluiu o ensino médio, que pode ser visualizada na \autoref{tab:tipo-ensino}; nem todos os alunos possuem essa informação. Pode-se observar que um valor significativo dos estudantes não têm o ano de conclusão informado (42,88\%), além de que aqueles que concluíram em ensino regular constituírem a maioria (19,15\%). Uma variável importante para essa análise da composição das turmas de ensino médio é CO\_ESCOLA. Ela representa um código que identifica de maneira unificada a instituição junto ao Ministério da Educação. É através dela que poderemos associar a qual escola esse estudante pertence dentro da base do Censo Escolar. De todos os estudantes, apenas 21.81\% possuem esse código.

  \begin{table}[h]
    \begin{tabular}{lcc}
    \hline
    \multicolumn{1}{c}{\textbf{Ano}} & \textbf{Quantidade} & \textbf{Percentual} \\ \hline
    2015                             & 966842              & 11,21\%             \\ \hline
    2014                             & 699987              & 8,11\%              \\ \hline
    2013                             & 527310              & 6,11\%              \\ \hline
    2012                             & 416454              & 4,83\%              \\ \hline
    2011                             & 317364              & 3,68\%              \\ \hline
    2010                             & 294214              & 3,41\%              \\ \hline
    2009                             & 244461              & 2,83\%              \\ \hline
    2008                             & 199619              & 2,31\%              \\ \hline
    2007                             & 178091              & 2,06\%              \\ \hline
    Anterior a 2007                  & 1083909             & 12,56\%             \\ \hline
    Não informado                    & 3699116             & 42,88\%             \\ \hline
    \end{tabular}
    \caption{Ano de conclusão de ensino médio dos candidatos do ENEM}
    \label{tab:ano-conclusao}
    \end{table}

    \begin{table}[h]
      \begin{tabular}{lcc}
      \hline
      \multicolumn{1}{c}{\textbf{Tipo de ensino}} & \multicolumn{1}{c}{\textbf{Quantidade}} & \multicolumn{1}{c}{\textbf{Percentual}} \\ \hline
      Ensino Regular                              & 1652485                                 & 19,15\%                                 \\ \hline
      Educação Especial - Modalidade Substitutiva & 10295                                   & 0,12\%                                  \\ \hline
      Educação de Jovens e Adultos                & 218532                                  & 2,53\%                                  \\ \hline
      \end{tabular}
      \caption{Tipo de ensino da escola dos candidatos do ENEM}
      \label{tab:tipo-ensino}
      \end{table}

Como nem todos os registros presentes no conjunto são relevantes para a pesquisa, estabelecemos alguns critérios para a aplicação de filtros.
\begin{itemize}
  \item Registros que possuam informação de gênero, já que sem ela não podemos analisar o efeito da composição de gênero;
  \item Registros de estudantes que já concluíram o ensino médio ou que irão concluir em 2016, excluíndo aqueles que concluirão após 2016 ou não estão cursando e não concluíram; 
  \item Registros de estudantes que tenham concluído o ensino médio após 2007, já que não há especificação do ano anterior a 2007;
  \item Registros de estudantes que realizaram ensino médio no Brasil, excluindo aqueles que estudaram no exterior por não haver informação dessas escolas;
  \item Registros de estudantes que não estão realizando o exame como treineiros, já que treineiros não estão aptos a utilizar a nota do ENEM para ingresso em universidade;
  \item Registros que possuam informação da escola de ensino médio, já que sem ela não é possível identificar a escola em outra base;
  \item Registros que possuam informação de todas as notas.
\end{itemize}

A \autoref{tab:filtros-enem} descreve as observações durante a aplicação dos filtros. Cada filtro foi aplicado sequencialmente, sendo que os valores mostrados correspondem ao total do conjunto original subtraindo os total de registros excluídos até o momento. Após a aplicação de todos os filtros, restaram-se 1388044 registros, que correspondem a 16,09\% do conjunto original. Esse subconjunto de dados filtrados do original é o que passa a ser utilizado nas análises subsequentes.

\begin{table}[h]
  \begin{tabular}{lcc}
  \hline
  \multicolumn{1}{c}{\textbf{Filtro}}     & \textbf{Quantidade} & \textbf{Percentual} \\ \hline
  Conjunto original                       & 8627367             & 100,00\%            \\ \hline
  Com informação de gênero                & 8627367             & 100,00\%            \\ \hline
  Já concluíram ou concluirão EM em 2016  & 6810529             & 78,94\%             \\ \hline
  Concluíram após 2007                    & 5726620             & 66,38\%             \\ \hline
  Concluíram EM no Brasil                 & 5725635             & 66,37\%             \\ \hline
  Concluíram EM no ensino regular         & 5496816             & 63,71\%             \\ \hline
  Não são treineiros                      & 5496816             & 63,71\%             \\ \hline
  Com informação presente da escola de EM & 1652471             & 19,15\%             \\ \hline
  Com informação presente de notas        & 1388044             & 16,09\%             \\ \hline
  \end{tabular}
  \caption{Filtros aplicados no conjunto de dados do ENEM}
  \label{tab:filtros-enem}
  \end{table}

Já no \textit{dataset} do SISU, os dados estão estruturados de forma diferente, com uma listagem (ou abertura) por semestre na realização do processo seletivo. Cada registro corresponde a uma inscrição de um estudante em um curso, totalizando 6665892 registros. Na \autoref{tab:inscricoes-inscritos-sisu}, podemos observar a quantidade de inscrições e participantes inscritos, divididos por listagem. Em ambos os casos, a primeira listagem concentra a maioria dos registros.

\begin{table}[h]
  \begin{tabular}{ccc}
  \hline
  \multicolumn{1}{c}{\textbf{Listagem}} & \textbf{Inscrições} & \textbf{Inscritos} \\ \hline
  1                                     & 4868545                           & 2494173                          \\ \hline
  2                                     & 1797347                           & 935538                           \\ \hline
  Total                                 & 6665892                           & 3429711                          \\ \hline
  \end{tabular}
  \caption{Inscrições e inscritos no SISU, divididos por listagem}
  \label{tab:inscricoes-inscritos-sisu}
  \end{table}

Quando segmentado pela opção de cursos nos quais o candidato se inscreveu, que pode ser de apenas 1 ou 2 cursos, pode-se perceber que 94,3\% dos participantes opta por se inscrever nas duas opções disponíveis, como mostra a \autoref{tab:quantidade-cursos-sisu}.

  \begin{table}[h]
    \begin{tabular}{lc}
    \hline
    \textbf{Opção de curso} & \textbf{Inscritos} \\ \hline
    Apenas 1 curso          & 193530                           \\ \hline
    2 cursos                & 3236181                          \\ \hline
    Total                   & 3429711                          \\ \hline
    \end{tabular}
    \caption{Quantidade de inscritos por cursos optados no SISU}
    \label{tab:quantidade-cursos-sisu}
    \end{table}

\pagebreak

É possível que um candidato realize inscrição em somente uma das aberturas ou em ambas, observado na \autoref{tab:caso-inscricao-sisu}. Mais da metade (64,4\%) optou por se inscrever apenas na primeira abertura.

\begin{table}[h]
  \begin{tabular}{lc}
  \hline
  \multicolumn{1}{c}{\textbf{Caso de inscrição}}        & \textbf{Inscritos} \\ \hline
  Apenas na listagem 1  & 1689691                          \\ \hline
  Apenas na listagem 2  & 131056                           \\ \hline
  Ambas as listagens & 804482                           \\ \hline
  \multicolumn{1}{c}{Total}                & 2625229                          \\ \hline
  \end{tabular}
  \caption{Quantidade de inscritos por caso de inscrição em listagem no SISU}
  \label{tab:caso-inscricao-sisu}
  \end{table}

Também é possível que o candidato tenha até 4 inscrições em cursos ao longo do ano, com o máximo de 2 por semestre. A \autoref{tab:total-inscricoes-inscritos-sisu} mostra a quantidade de inscritos pelo total de inscrições. A maior parte dos estudantes realizou 2 inscrições (65,08\%), com o segundo maior total sendo de 4 inscrições (27.90\%). 

\begin{table}[]
  \centering
  \begin{tabular}{cc}
  \hline
  \textbf{Total de inscrições} & \textbf{Inscritos} \\ \hline
  1                   & 116802    \\ \hline
  2                   & 1708599   \\ \hline
  3                   & 67420     \\ \hline
  4                   & 732408    \\ \hline
  Total               & 2625229   \\ \hline
  \end{tabular}
  \caption{Quantidade de inscritos pelo total de inscrições no SISU}
  \label{tab:total-inscricoes-inscritos-sisu}
  \end{table}

De forma semelhante ao caso do ENEM, foram realizados alguns filtros no \textit{dataset} do SISU, sendo eles a exclusão de registros que não possuam informação de gênero e de registros que não possuam a informação de todas as notas. Não houve retirada de nenhum registro após a aplicação dos filtros, já que essas informações estavam disponíveis em todos eles. Também foi aplicada uma técnica de transformação de dados, de modo a agrupar todas as inscrições de um candidato em um único registro. Assim, houve uma redução de 60.62\% no \textit{dataset}, passando de 6665892 a 2625230 registros. Agora, o registro deixa de ser a inscrição de um candidato em um único curso para ser a inscrição no SISU como um todo.

Para identificar cada participante unicamente nas bases de dados, é necessário atribuir um campo como chave identificadora. Boas chaves candidatas para esse propósito seriam o CPF ou número de inscrição. Diferente de outros estudos da literatura, estamos lidando com dados disponibilizados publicamente. Isso significa que informações sensíveis, tais como essas chaves, não foram fornecidas. Assim, foi necessário desenvolver uma nova chave a partir dos campos existentes. 
Para isso, realizamos uma investigação nas notas das provas, que foram agrupadas 
para observar se é possível utilizá-las como identificador único. 

No \textit{dataset} do ENEM, as notas agrupadas foram suficientes para identificar cada um dos alunos. Isso porque não há valores duplicados para essas notas, ou seja, cada aluno possui notas únicas. No SISU, isso não foi possível de ser observado, já que 22 notas estão repetidas. Essas notas pertencem a 75 inscritos. As notas repetidas se enquadram em dois casos: 4 notas objetivas zeradas, com exceção da redação (ex.: 0 - 0 - 0 - 0 - 520) ou todas as notas maiores que zero (ex.: 570,4 - 619,6 - 433,5 - 546,1 - 520). Não há caso de candidatos com todas as notas iguais a zero, pois o SISU impede a inscrição quando a nota da redação é zerada. Então, utilizamos outros campos para compor o identificador. O identificador final constitui-se das notas agrupadas, idade, unidade federativa de residência e gênero. Como esses campos estão disponíveis em ambas as bases, é possível identificar o mesmo candidato nas bases com a mesma chave identificadora.

Os cursos de graduação foram agrupados de acordo com a área do conhecimento, utilizando a classificação documentada pelo International Standard Classification of Education\footnote{https://uis.unesco.org/sites/default/files/documents/international-standard-classification-of-education-isced-2011-en.pdf}, realizada pela UNESCO. Nessa classificação, 25 áreas do conhecimento, e seus respectivos cursos, estão organizadas em 9 grupos: 

\begin{itemize}
  \item Educação;
  \item Humanidades e artes;
  \item Ciências sociais, negócios e direito;
  \item Ciência;
  \item Engenharia, manufatura e construção;
  \item Agricultura;
  \item Saúde e bem-estar;
  \item Serviços;
  \item Não informado.
\end{itemize}

Através do identificador criado, foi realizada a combinação das bases de dados. O resultado foi uma base única que associa todas as informações dos candidatos presentes nos \textit{datasets} do ENEM e SISU. Nessa base, são encontrados 712526 registros, que representam participantes encontrados tanto no SISU, quanto no ENEM. 27,14\% dos participantes do ENEM foram encontrados no SISU, ao passo que 51,33\% dos participantes do SISU foram encontrados no ENEM. Vale notar que, para se inscrever no SISU, é obrigatório a realização do ENEM, mas nesse estudo, estamos utilizando apenas uma parte dos dados do exame, por isso a correspondência de participantes do SISU no ENEM não é de 100\%.

O conjunto de dados obtido será utilizado na combinação com uma terceira fonte, a do Censo Escolar. Através do Censo, poderemos caracterizar as escolas de ensino médio nas quais os estudantes se formaram. Os dados dessa fonte passarão por pré-processamento e análise exploratória, semelhante ao já realizado. Também com esses dados, realizaremos a aplicação da metodologia de variações idiossincráticas na composição de gênero através das coortes escolares. Isso será realizado em etapas futuras da pesquisa, que serão descritas no \autoref{cap:plano-trabalho}. O próximo capítulo apresentará os resultados da análise descritiva dos dados levantados até então. Essa análise está segmentada por gênero e por aprovação nos cursos.