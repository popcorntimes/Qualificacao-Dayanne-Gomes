\chapter{Trabalhos relacionados}

Um estudo de grande relevância para o entendimento de como os pares afetam os colegas na sala de aula é o de \citet{Hoxby-2000}. Nessa pesquisa, a economista Caroline Hoxby explora como a composição de uma turma escolar pode influenciar a experiência de aprendizado dos estudantes e suas conquistas acadêmicas. São utilizados dados de alunos do 4° ao 7° ano do ensino fundamental de escolas públicas do Texas, Estados Unidos na década de 1990. A autora aborda duas fontes de variação idiossincrática, sendo elas as mudanças na composição de gênero e raça de uma turma escolar em anos adjacentes. Duas estratégias empíricas discutem essas variações: na primeira, são avaliados os efeitos de ter um grupo com maioria feminina e diferente composição racial; na segunda, são avaliados os efeitos das conquistas dos pares em grupos masculinos e femininos. Os resultados de Hoxby sugerem que ter uma maioria de colegas do gênero feminino aumenta a pontuação de meninos e meninas em leitura e matemática, que pode ser devido a fenômenos como a menor tendência de mulheres causarem disrupções nas salas de aula.

De forma similar, outros trabalhos estimam os efeitos da composição de gênero através de variações idiossincráticas. \citet{Schne2019} visaram entender como a composição de gênero afeta escolhas educacionais. Para isso, foram utilizados dados de alunos dos anos finais do ensino fundamental da Noruega, entre os anos de 2003 e 2008. Nesse cenário, são considerados os efeitos da parcela feminina de pares nas escolhas de áreas e disciplinas do ensino médio, mais especificamente naqueles orientados a STEM. Os autores observam que um aumento na proporção de meninas torna tanto meninos quanto meninas mais propensos a escolher mais disciplinas da área STEM e menos disciplinas de linguagens no ensino médio. Também sugere-se que uma maioria feminina tem um efeito positivo no desempenho de garotas em matemática, que pode impactar na probabilidade de escolher mais cursos de STEM e diminuir a discriminação de gênero sentida por elas. 

\citet{Lavy2011} investigam os efeitos da composição de gênero em diferentes estágios da trajetória escolar, além do desempenho acadêmico de meninos e meninas. Para isso, são utilizados dados de estudantes de Israel do ensino fundamental, entre os anos de 2002 a 2005, e médio, de 1993 a 2000. São utilizados como resultados as notas de disciplinas e performance em exames de entrada. Alguns dos mecanismos destacados são a disrupção e violência na sala de aula, interação entre estudantes, relacionamento aluno-professor e senso de fatiga de professores na sua profissão. Evidências apontam que o aumento de proporções de garotas, em especial quando elas são maioria na coorte, levam a melhores ambientes escolares e aprendizagem. Os benefícios são ainda maiores para aqueles em contexto socioeconômico vulnerável, com grandes impactos entre estudantes imigrantes e que tenham pais com baixa escolaridade.

\citet{Schneeweis2012} identificam o impacto causal da composição de gênero na escolha de um campo acadêmico. Os dados cobrem os anos de 1988 e 2006 da cidade de Linz, Áustria, do ensino fundamental, com foco no último ano. No contexto austríaco, os estudantes podem escolher uma área dentro de uma escola vocacional, que prepara para uma vaga de trabalho, ou seguir estudos superiores na universidade. Além de levar em consideração o interesse vocacional, também é estimado o impacto de estudar em turmas com mais colegas do gênero feminino, que pode levar a escolhas de áreas mais técnicas, a depender da composição. Os resultados desse estudo mostram que estudantes do gênero feminino são menos prováveis de escolher áreas com maioria feminina e mais prováveis de seguir áreas técnicas quando expostas a maiores proporções femininas.

No trabalho de \citet{Brene2020} é observado o efeito dos pares a longo prazo. A utilização de dados de registros estudantis que ingressam na área de matemática no ensino médio entre os anos de 1980 e 1994 possibilitou o acompanhamento dessa população pelo período de 20 anos. Assim, a investigação dos efeitos da composição de gênero na participação na área STEM na Dinamarca pode acompanhar não só as escolhas educacionais, mas as consequências diretas e retardadas ao longo do tempo em turmas de ensino médio. São observadas probabilidade de entrada e finalização de um curso STEM, bem como os ganhos salariais e propensão de homens e mulheres terem filhos em diferentes etapas da vida. Brenøe e Zölitz observam que maiores proporções de colegas do gênero feminino no ensino médio tornam as escolhas são mais esteriotipadas, ou seja, pessoas de um gênero escolhem áreas mais associadas ao seu gênero. Nesse cenário, mulheres escolhem mais áreas da saúde, ao passo que homens escolhem mais áreas STEM. Além disso, observações a longo prazo indicam uma menor probabilidade de mulheres trabalharem em cargos STEM, além de terem uma menor remuneração. Elas também se caracterizam de forma diferente aos homens com relação à decisão de ter filhos.

Ainda sobre estudos que observam variações idiossincráticas na composição de gênero, temos um recente trabalho que considera o contexto nacional. O trabalho de \citet{Borges2021} utiliza essa metodologia para avaliar se a composição de gênero de coortes do ensino médio influencia a escolha de curso de graduação de estudantes, em especial para mulheres. Essa análise foi realizada com dados do vestibular entre os anos 2000 a 2008 de uma universidade pública, a UNICAMP, que foram relacionados aos dados do Censo Escolar. Borges observou que a porcentagem de pares femininos impacta na escolha de curso, diferindo para homens e mulheres. Uma maior proporção de colegas do gênero feminino reduz a probabilidade da escolha de cursos focados em matemática ou ciência para ambos os gêneros. A autora também verifica que, para mulheres, essa exposição aumenta a escolha de cursos competitivos, com maiores notas de corte. Candidatas expostas a menores proporções de pares femininos ($\le33\%$) escolhem cursos menos competitivos.

Outros estudos também utilizaram dados educacionais brasileiros para analisar diferentes fatores. \citet{Machado2021} utilizaram dados entre os anos de 2010 e 2017 do SISU e do Censo Escolar para investigar os impactos de sistemas de admissão centralizados na composição de estudantes. As autoras observaram características dos estudantes como gênero, idade, etnia e migração, além de características das escolas para mensurar os efeitos do SISU na atração de candidatos de diversos perfis. \citet{Mello2022} analisa como reformas educacionais que expandiram o acesso à educação superior impactaram a admissão de estudantes de baixa renda. Essas políticas incluem a expansão da centralização de aplicações com o SISU e mais oferta de cotas de ações afirmativas. São utilizados dados do Censo da Educação Superior dos anos de 2010 a 2015 e do ENEM dos anos de 2009 a 2014. \citet{Otero2021} também conduziram um estudo sobre as consequências de ações afirmativas no contexto da admissão em universidades brasileiras. Eles exploraram questões como escolhas de área, frequência e persistência na universidade e rendimentos projetados. Para tal, são utilizados dados do ENEM de 2009 a 2015, SISU de 2016, Censo da Educação Superior entre 2009 e 2019 e Relação Anual de Informações Sociais de 2017. \citet{Carvalhaes2022} investigaram a interseção entre renda e raça na estruturação do acesso ao ensino superior. São utilizados dados de estudantes concluintes do ensino médio entre 2012 e 2017 para estimar a probabilidade de pessoas de diferentes contextos ingressarem no ensino superior, bem como a decomposição de efeitos diretos e indiretos de renda e raça. 

Existem diversos trabalhos na literatura que abordam como múltiplos fatores influenciam jovens inseridos no contexto educacional, sendo um deles o efeito da composição de gênero. Dentre aqueles que considerassem o contexto brasileiro, porém, foram encontrados poucos estudos que focassem nesse fator específico. Entre os trabalhos apresentados, o trabalho de \citet{Borges2021} é o que mais se assemelha ao que estamos fazendo, porém seus dados se restringem a uma universidade em particular. Nossa motivação é basear-se na metodologia de variações idiossincráticas na composição de gênero entre coortes de uma mesma escola para expandir a análise ao nível nacional, utilizando dados do ENEM, SISU e Censo Escolar para observar como estudantes realizam suas escolhas de curso superior em todo o país e como elas são influenciadas por seus pares no ensino médio. 