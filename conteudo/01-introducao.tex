\chapter{Introdução}

O momento da escolha de carreira profissional é um acontecimento de grande importância na vida de um indivíduo. Esse processo é determinante na definição de características individuais e coletivas de pessoas que se veem na responsabilidade de tomar uma decisão impactante \autocite{AkosahTwumasi2018} cujas consequências podem influenciar uma vida inteira. As implicações de optar por um determinado caminho, tanto no âmbito acadêmico como profissional, vão além da preferência por uma área do conhecimento. Elas podem influenciar de forma significativa a qualidade de vida futura, já que sua ocupação tem poder de influenciar personalidade, nível de renda, status social e grupos sociais nos quais os sujeitos se caracterizarão \autocite{Kazi2017}.

As mudanças ocasionadas pelas transformações sociais fizeram com que se percebesse a necessidade de entender como a juventude é afetada no mundo contemporâneo \autocite{Unesco2006}. A responsabilidade de tomar decisões significativas não é alheia a esses jovens. Segundo \citet{Gati2001}, adolescentes estão envolvidos diretamente no processo de escolhas complexas. Preocupações com questões educacionais, como estudos e carreira, mostram um nível de entendimento dos riscos e implicações envolvidos, além das dificuldades que podem surgir.

Esses temas podem ser observados de forma muito expressiva em estudantes do ensino médio. Para aqueles que estão vivendo o término da adolescência e início da vida adulta, a transição envolve não só o amadurecimento etário, mas também a passagem da educação secundária para a superior. Para uma grande maioria desses estudantes, há a aspiração de adentrar algum curso de graduação \autocite{Venezia2013}. A escolha de frequentar a universidade pode começar cedo durante a jornada educacional e, nesse caminho até adentrar o curso desejado, existem diversas etapas \autocite{Cabrera2000}. Em todo o mundo, exames educacionais competitivos são aplicados para medir o desempenho de alunos secundaristas. Os resultados desses exames podem ser utilizados para ranquear os melhores candidatos e convocá-los para admissão em uma institução. Alguns exemplos são \href{https://www.act.org/content/act/en/products-and-services/the-act.html}{ACT} e  \href{https://satsuite.collegeboard.org/sat}{SAT}, nos Estados Unidos, \href{https://www.education.gouv.fr/reussir-au-lycee/le-baccalaureat-general-10457}{Baccalauréat}, na França e \href{https://www.britishcouncil.es/en/exam/professional-university/igcse}{GCSE}, no Reino Unido.

No Brasil, tais exames são conhecidos como vestibular. Os participantes costumam realizá-lo no último ano do ensino médio, quando estão encerrando seus estudos, ou até que consigam a classificação para a vaga pretendida. As instituições têm autonomia para adotar uma prova específica, como o vestibular tradicional, cuja aplicação é realizada pela própria instituição. Outra opção é a adoção do \href{https://www.gov.br/inep/pt-br/areas-de-atuacao/avaliacao-e-exames-educacionais/enem}{ENEM} como forma de entrada. O Exame Nacional do Ensino Médio é uma prova anual com abrangência em todo território brasileiro. A nota desse exame, além de avaliar o desempenho de estudantes secundaristas, pode ser utilizada para adentrar algum curso superior através de programas governamentais, como SISU, ProUni e FIES.

Dadas diversas formas de ingresso, ainda é parte da escolha do estudante qual curso de graduação ele deseja seguir. Múltiplos fatores devem ser considerados nesse processo. Para \citet{Borchert2001}, existem três grandes perpectivas que afetam a escolha de carreira: oportunidade, personalidade e ambiente. Segundo \citet{Abbagnano2012}, ambiente pode ser definido como um complexo conjunto de relações entre mundo natural e ser vivo, que influem na vida e no comportamento do indivíduo. O gênero tem um grande papel nas questões relacionadas ao ambiente no qual os jovens se encontram. No cenário brasileiro, as questões de gênero são relevantes e consideradas no processo de levantamento de indicadores sociais. Um exemplo é o estudo \href{https://biblioteca.ibge.gov.br/visualizacao/livros/liv101784_informativo.pdf}{Estatísticas de gênero: indicadores sociais das mulheres no Brasil}, do Instituto Brasileiro de Geografia e Estatística (IBGE). Nele, são levantados diferentes aspectos da vida da população, incluindo desigualdades no mercado de trabalho e na educação enfrentadas pelas mulheres. Apesar haver avanços recentes na igualdade de participação, elas ainda sofrem com uma subrepresentatividade em áreas com predominância masculina, como Ciência, Tecnologias, Engenharias e Matemáticas (STEM) \autocite{Saavedra2010}.que afeta a diversidade de gênero das áreas e interfere no . 

\section{Motivação e Objetivos}
Neste contexto, queremos descrever o perfil dos estudantes, em especial de mulheres, e entender como um fator específico, o gênero, impacta na decisão de carreira. Para o processo de análise das escolhas de um curso de graduação, em particular cursos com histórica baixa representatividade feminina, bem como a composição de turmas do ensino médio, levantamos as seguintes questões de pesquisa:

\begin{itemize}
  \item \textbf{Q1} Há diferença no desempenho de homens e mulheres nas disciplinas do ENEM?
  \item \textbf{Q2} Homens e mulheres que realizam o processo seletivo do SISU fazem escolhas distintas? Em quais cursos estão mais concentrados? Há uma diferença na proporção daqueles que selecionam cursos STEM?
  \item \textbf{Q3} Quais são as características das escolas de ensino médio nas quais os estudantes se formaram?
  \item \textbf{Q4} A composição de gênero da coorte do ensino médio influencia a escolha de graduação das estudantes? Mulheres em escolas com mais colegas do sexo feminino são mais ou menos propensas a selecionaram cursos STEM? Há um efeito sobre a probabilidade das mulheres escolherem o curso de ciência da computação?
\end{itemize}

Para alcançar os objetivos dessa pesquisa, realizaremos um estudo de coortes levando em consideração os efeitos da composição de gênero. Serão aplicadas técnicas de ciência de dados para alcançar os objetivos deste trabalho. A ciência de dados é uma área multidisciplinar que envolve técnicas estatísticas, matemáticas e computacionais para resolver problemas de um determinado domínio. Os dados utilizados nesses processo passam uma série de etapas, como tratamento, análise e visualização, a fim de dar suporte à tomada de decisões estratégicas. Trabalharemos com dados educacionais a nível nacional de múltiplas fontes, incluindo ENEM, SISU e Censo Escolar.


