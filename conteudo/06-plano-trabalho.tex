\chapter{Plano de trabalho}
\label{cap:plano-trabalho}

Nos capítulos anteriores, exploramos dois conjuntos de dados educacionais brasileiros, do ENEM e SISU, para entender como os estudantes estão caracterizados. Alguns aspectos levantados desse processo estão diretamente ligados com os objetivos desse trabalho, como as notas do exame, que refletem o desempenho dos participantes, e escolhas de curso e instituição de ensino, que variam de acordo com a segmentação. Para trabalhar com ambos os conjuntos numa visão única, desenvolvemos um novo identificador, que permite associar os registros e agregar todas as informações disponibilizadas apenas com campos públicos. 

Apesar disso, não foi possível obter um panorama detalhado das escolas de ensino médio desses jovens, já que essas informações estão disponíveis no conjunto de dados do Censo Escolar, que ainda não foi explorado. Sem esses dados, também não é possível a aplicação do método de variações idiossincráticas na composição de gênero, conforme apresentado no \autoref{sec:variacoes}, já que ele precisa das informações das coortes de ensino médio. Então, para que possamos dar continuidade à pesquisa, levantamos as seguintes atividades:

\begin{itemize}
    \item Levantamento das escolas dos estudantes presentes no conjunto ENEM/SISU através do código INEP;
    \item Realização de pré-processamento e análise exploratória dos dados do Censo Escolar, a fim de apresentar as características das escolas;
    \item Aplicação do método de variações idiossincráticas na composição de gênero, a fim de entender o efeito dos pares na escolha de graduação dos estudantes.
\end{itemize}

  Realizaremos um refinamento dos resultados levantados até então, finalizando a exploração dos dados do ENEM e SISU, conforme levantado pelas duas primeiras questões de pesquisa (Q1 e Q2). Após a conclusão desta etapa, serão executadas as tarefas relacionadas aos dados do Censo Escolar para realização do estudo do efeito da composição de gênero, com o intuito de responder as questões de pesquisa restantes (Q3 e Q4). A \autoref{tab:cronograma} apresenta o cronograma das atividades restantes. Pretende-se realizar a defesa da dissertação em Agosto de 2024.

\begin{table}[]
    \centering
    \begin{tabular}{ccccccccc}
    \hline
    Atividades                                                                                                            & Jan & Fev & Mar & Abr & Mai & Jun & Jul & Ago \\ \hline
    \begin{tabular}[c]{@{}c@{}}Refinamento de resultados e\\ levantamento de escolas\end{tabular}                         & X   &     &     &     &     &     &     &     \\ \hline
    \begin{tabular}[c]{@{}c@{}}Pré-processamento e análise\\ exploratória dos dados do\\ Censo Escolar\end{tabular}       &     & X   & X   &     &     &     &     &     \\ \hline
    \begin{tabular}[c]{@{}c@{}}Aplicação do método de variações\\ idiossincráticas na composição de\\ gênero\end{tabular} &     &     &     & X   & X   & X   &     &     \\ \hline
    Escrita da dissertação                                                                                                &     &     &     &     & X   & X   & X   & X   \\ \hline
    Defesa da dissertação                                                                                                    &     &     &     &     &     &     &     & X   \\ \hline
    \end{tabular}
    \caption{Cronograma de atividades (Jan 2024 - Ago 2024)}
    \label{tab:cronograma}
    \end{table}