%!TeX root=../tese.tex
%("dica" para o editor de texto: este arquivo é parte de um documento maior)
% para saber mais: https://tex.stackexchange.com/q/78101/183146

% As palavras-chave são obrigatórias, em português e em inglês, e devem ser
% definidas antes do resumo/abstract. Acrescente quantas forem necessárias.
\palavrachave{Ciência de dados}
\palavrachave{Composição de gênero}
\palavrachave{Escolhas de graduação}
\palavrachave{Ensino médio}
\palavrachave{Educação}
\palavrachave{SISU}
\palavrachave{ENEM}
\palavrachave{Censo Escolar}

\keyword{Data science}
\keyword{Gender composition}
\keyword{Major choices}
\keyword{High school}
\keyword{Education}
\keyword{SISU}
\keyword{ENEM}
\keyword{School Census}

% O resumo é obrigatório, em português e inglês. Estes comandos também
% geram automaticamente a referência para o próprio documento, conforme
% as normas sugeridas da USP.
\resumo{
    A ciência de dados é uma área interdisciplinar que combina técnicas computacionais, estatísticas e matemáticas para analisar dados de um determinado domínio a fim de extrair informações relevantes. Despontando em diferentes setores da sociedade, ela é uma aliada poderosa do processo de tomada de decisão, já que permite a extração de informação de grandes conjuntos de dados e transformá-la em conhecimento. Assim, esse conjunto de ferramentas pode nos ajudar a descrever um aspecto da educação no cenário brasileiro: a escolha do curso de graduação. O objetivo desta pesquisa de mestrado é utilizar ciência de dados para investigar como a composição de gênero da coorte do ensino médio influencia a escolha do curso superior de jovens estudantes. Para isso, buscamos entender as diferenças de desempenho entre homens e mulheres, bem como a concentração e proporção em cursos específicos. Além disso, visamos caracterizar as escolas de origem dos estudantes e as propensões de jovens do gênero feminino a escolherem cursos com baixa representatividade feminina, especificamente na área de ciências exatas e engenharias. Para tal, utilizamos dados escolares coletados de fontes públicas, tais como ENEM, SISU e Censo Escolar, para alimentar o processo de ciência de dados, empregando técnicas que incluem pré=processamento, análise exploratória e variações idiossincráticas na composição de gênero, a fim de extrair insights valiosos dos dados disponíveis. Essa pesquisa traz como contribuição à literatura a combinação de múltiplas bases de dados nacionais para o levantamento de identificadores, que pode auxiliar na orientação de esforços para criar ambientes mais inclusivos e diversificados na educação superior.
}

\abstract{
    Data science is an interdisciplinary field that combines computational, statistical, and mathematical techniques to analyze data from a specific domain to extract relevant information. Emerging in different sectors of society, it is a powerful ally in the decision-making process, as it enables the extraction of information from large datasets and transforms it into knowledge. Thus, this set of tools can help us describe an aspect of education in the Brazilian scenario: the choice of undergraduate courses. This master's research aims to use data science to investigate how the gender composition of the high school cohort influences the choice of higher education courses by young students. To achieve this, we seek to understand performance differences between men and women, as well as the concentration and proportion in specific courses. Additionally, we aim to characterize the origin schools of students and the inclinations of young females to choose courses with low female representation, specifically in the fields of exact sciences and engineering. For this purpose, we use school data collected from public sources such as ENEM, SISU, and School Census to feed the data science process, employing techniques that include preprocessing, exploratory analysis, and idiosyncratic variations in gender composition, to extract valuable insights from the available data. This research contributes to the literature by combining multiple national databases to gather identifiers, which can assist in guiding efforts to create more inclusive and diverse environments in higher education.
}
