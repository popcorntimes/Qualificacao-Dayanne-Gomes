\chapter{Trabalhos relacionados}

Um estudo de grande relevância para o entendimento de como os pares afetam os colegas na sala de aula é o de \citet{Hoxby-2000}. Nessa pesquisa, a economista Caroline Hoxby explora como a composição de uma turma escolar pode influenciar a experiência de aprendizado dos estudantes e suas conquistas acadêmicas. São utilizados dados de alunos do 4° ao 7° ano do ensino fundamental de escolas públicas do Texas, Estados Unidos na década de 1990. A autora aborda duas fontes de variação idiossincrática, sendo elas as mudanças na composição de gênero e raça de uma turma escolar em anos adjacentes. Duas estratégias empíricas discutem essas variações: na primeira, são avaliados os efeitos de ter um grupo com maioria feminina e diferente composição racial; na segunda, são avaliados os efeitos das conquistas dos pares em grupos masculinos e femininos.

De forma similar, outros trabalhos abordam os efeitos da composição de gênero através de variações idiossincráticas. \citet{Schne2019} visaram entender como a composição de gênero afeta escolhas educacionais. Para isso, foram utilizados dados de alunos dos anos finais do ensino fundamental da Noruega, entre os anos de 2003 e 2008. Nesse cenário, são considerados os efeitos da parcela feminina de pares nas escolhas de áreas e disciplinas do ensino médio, mais especificamente naqueles orientados a STEM. Algumas das hipóteses levantadas investigam se há efeitos na performance escolar, na perpetuação de estereótipos de gênero e na competitividade.

\citet{Lavy2011} apresentam a extensão dos efeitos da composição de gênero na função de produção educacional. Nesse trabalho, são investigados as conquistas educacionais de meninos e meninas em diferentes estágios do sistema educacional. Para isso, são utilizados dados de estudantes de Israel do ensino fundamental, entre os anos de 2002 a 2005, e médio, de 1993 a 2000. São utilizados como resultados as notas de disciplinas e performance em exames de entrada. Alguns dos mecanismos destacados são a disrupção e violência na sala de aula, interação entre estudantes, relacionamento aluno-professor e senso de fatiga de professores na sua profissão.

Outro aspecto considerado pela literatura é a diferença no impacto de escolas segregadas por sexo e coeducacionais, onde meninos e meninas são ensinados juntos, como no  trabalho de \citet{Schneeweis2012}. Schneeweis e Zweimüller identificam o impacto causal da composição de gênero na escolha de um campo acadêmico. Os dados cobrem os anos de 1988 e 2006 da cidade de Linz, Áustria, do ensino fundamental, com foco no último ano. No contexto austríaco, os estudantes podem escolher uma área dentro de uma escola vocacional, que prepara para uma vaga de trabalho, ou seguir estudos superiores na universidade. Além de levar em consideração o interesse vocacional, também é estimado o impacto de estudar em turmas com mais colegas do gênero feminino, que pode levar a escolhas de áreas mais técnicas, a depender da composição.

No trabalho de \citet{Brene2020} é observado o efeito dos pares a longo prazo. A utilização de dados de registros de estudantes que ingressam na linha de matemática no ensino médio entre os anos de 1980 e 1994 possibilitou o acompanhamento de toda essa população pelo período de 20 anos. Assim, a investigação dos efeitos da composição de gênero na participação na área STEM na Dinamarca pode acompanhar não só as escolhas educacionais, mas as consequências diretas e retardadas ao longo do tempo em turmas de ensino médio. Leva-se em consideração se a exposição a mais pares femininos está correlacionada com a disparidade de gênero. São observadas probabilidade da entrada e finalização de um curso STEM, bem como os ganhos salariais e a fertilidade de homens e mulheres em diferentes etapas da vida.

Ainda sobre estudos que observam variações idiossincráticas na composição de gênero, temos um recente trabalho que considera o contexto nacional. O trabalho de \citet{Borges2021}