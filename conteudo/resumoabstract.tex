%!TeX root=../tese.tex
%("dica" para o editor de texto: este arquivo é parte de um documento maior)
% para saber mais: https://tex.stackexchange.com/q/78101/183146

% As palavras-chave são obrigatórias, em português e em inglês, e devem ser
% definidas antes do resumo/abstract. Acrescente quantas forem necessárias.
\palavrachave{Ciência de dados}
\palavrachave{Composição de gênero}
\palavrachave{Escolhas de graduação}
\palavrachave{Ensino médio}
\palavrachave{Educação}
\palavrachave{SISU}
\palavrachave{ENEM}
\palavrachave{Censo Escolar}

\keyword{Data science}
\keyword{Gender composition}
\keyword{Major choices}
\keyword{High school}
\keyword{Education}
\keyword{SISU}
\keyword{ENEM}
\keyword{School Census}

% O resumo é obrigatório, em português e inglês. Estes comandos também
% geram automaticamente a referência para o próprio documento, conforme
% as normas sugeridas da USP.
\resumo{
    Dentro do contexto educacional, existe um momento decisivo na trajetória de jovens estudantes do ensino médio: a escolha do curso de graduação. O ingresso no ensino superior leva em consideração múltiplos fatores determinantes, dentre eles, as questões de gênero. Mulheres enfrentam diferentes desafios, sendo alvos de disparidade de gênero dentro de suas carreiras acadêmicas e profissionais. Alguns estudos existentes na literatura abordam como a exposição a mais pares do gênero feminino pode impactar nas escolhas de meninos e meninas, mas ainda há poucos estudos que abordam o cenário brasileiro. O objetivo desta pesquisa de mestrado é utilizar ciência de dados para investigar como a composição de gênero da coorte do ensino médio influencia a escolha do curso superior de jovens estudantes no Brasil. Para isso, utilizaremos dados do Exame Nacional do Ensino Médio (ENEM), combinando dados do Sistema de Seleção Unificada (SISU), utilizando metodologias econométricas para análise do efeito. Em específico, utilizaremos variações idiossincráticas na composição de gênero entre coortes de uma mesma escola para estimação de um efeito causal. Além disso, visamos caracterizar diferenças de gênero no desempenho no ENEM, bem como caracterizar as escolas de origem dos estudantes e as propensões de jovens do gênero feminino a escolherem cursos com baixa representatividade feminina, especificamente na área de ciências exatas e engenharias. São utilizados dados educacionais públicos do ENEM, SISU e Censo Escolar para alimentar o processo de ciência de dados, empregando uma metodologia de pré-processamento e análise exploratória de dados, a fim de extrair percepções valiosas dos dados disponíveis. Nesta etapa preliminar do trabalho, foi realizada a combinação dos conjuntos de dados do ENEM e SISU através da criação de um identificador único. Isso possibilitou a caracterização do perfil dos candidatos, instituições de ensino superior e cursos de graduação, observando-se diferenças entre os gêneros. Na continuação deste trabalho, será realizada a exploração dos dados das escolas dos estudantes e futuramente a aplicação do método de variações idiossincráticas na composição de gênero.
}

\abstract{
    In the educational context, there is a decisive moment in the trajectory of young high school students: major choices. Admission to higher education takes into account multiple determining factors, including gender issues. Women face different challenges, being targets of gender disparity within their academic and professional careers. Some studies in the literature address how exposure to more female peers can impact the choices of boys and girls, but there are still few studies that address the Brazilian scenario. The objective of this master's research is to use data science to investigate how the gender composition of the high school cohort influences the higher education course choice of young students in Brazil. To do this, we will use data from the National Secondary Education Examination (ENEM), combining data from the Unified Selection System (SISU), using econometric methodologies to analyze the effect. Specifically, we will use idiosyncratic variations in gender composition between cohorts from the same school to estimate a causal effect. Furthermore, we aim to characterize gender differences in ENEM performance, as well as characterize the students' schools and the propensities of young females to choose majors with low female representation, specifically in the STEM field. Public educational data from ENEM, SISU, and School Census are used to feed the data science process, employing a pre-processing and exploratory data analysis methodology to extract valuable insights from the available data. In this preliminary stage of the work, the ENEM and SISU data sets were combined by creating a unique identifier. This made it possible to characterize the profile of candidates, higher education institutions, and majors, observing differences between genders. In the continuation of this work, data from students' schools will be explored and, in the future, the method of idiosyncratic variations in gender composition will be applied.
}
