%!TeX root=../tese.tex
%("dica" para o editor de texto: este arquivo é parte de um documento maior)
% para saber mais: https://tex.stackexchange.com/q/78101/183146

% As palavras-chave são obrigatórias, em português e em inglês, e devem ser
% definidas antes do resumo/abstract. Acrescente quantas forem necessárias.
\palavrachave{Ciência de dados}
\palavrachave{Composição de gênero}
\palavrachave{Escolhas de graduação}
\palavrachave{Ensino médio}
\palavrachave{Educação}
\palavrachave{SISU}
\palavrachave{ENEM}
\palavrachave{Censo Escolar}

\keyword{Data science}
\keyword{Gender composition}
\keyword{Major choices}
\keyword{High school}
\keyword{Education}
\keyword{SISU}
\keyword{ENEM}
\keyword{School Census}

% O resumo é obrigatório, em português e inglês. Estes comandos também
% geram automaticamente a referência para o próprio documento, conforme
% as normas sugeridas da USP.
\resumo{
    Dentro do contexto educacional, existe um momento decisivo na trajetória de jovens estudantes do ensino médio: a escolha do curso de graduação. O ingresso no ensino superior leva em consideração múltiplos fatores determinantes, dentre eles, as questões de gênero. Mulheres enfrentam diferentes desafios, sendo alvos de disparidade de gênero dentro de suas carreiras acadêmicas e profissionais. Alguns estudos existentes na literatura abordam como a exposição a mais pares do gênero feminino pode impactar nas escolhas de meninos e meninas em idade escolar, mas ainda há poucos estudos que abordam o cenário brasileiro. O objetivo desta pesquisa de mestrado é utilizar ciência de dados para investigar como a composição de gênero da coorte do ensino médio influencia a escolha do curso superior de jovens estudantes no Brasil. Para isso, buscamos entender as diferenças de desempenho no Exame Nacional do Ensino Médio (ENEM) entre homens e mulheres, bem como a concentração em áreas do conhecimento ao concorrerem a vagas através do Sistema de Seleção Unificada (SISU). Além disso, visamos caracterizar as escolas de origem dos estudantes e as propensões de jovens do gênero feminino a escolherem cursos com baixa representatividade feminina, especificamente na área de ciências exatas e engenharias. São utilizados dados educacionais públicos do ENEM, SISU e Censo Escolar para alimentar o processo de ciência de dados, empregando uma metodologia de pré-processamento e análise exploratória de dados, além de observar variações idiossincráticas na composição de gênero, a fim de extrair percepções valiosas dos dados disponíveis. Nesta etapa preliminar do trabalho, foi realizada a combinação dos conjuntos de dados do ENEM e SISU através da criação de um identificador único. Isso possibilitou a caracterização do perfil dos candidatos, instituições de ensino superior e cursos de graduação, observando-se diferenças entre os gêneros. Na continuação deste trabalho, será realizada a exploração dos dados das escolas dos estudantes para aplicação do método de variações idiossincráticas na composição de gênero.
}

\abstract{
    In the educational context, there is a decisive moment in the trajectory of young high school students: the choice of undergraduate courses. Admission to higher education takes into account multiple determining factors, among them, gender issues. Women face different challenges, being targets of gender disparity within their academic and professional careers. Some existing studies in the literature address how exposure to more peers of the female gender can impact the choices of boys and girls of school age, but few studies are exploring the Brazilian scenario. The objective of this master's research is to use data science to investigate how the gender composition of the high school cohort influences the choice of higher education courses for young students in Brazil. To achieve this, we seek to understand performance differences in the National High School Exam (ENEM) between men and women, as well as the concentration in areas of knowledge when competing for positions through the Unified Selection System (SISU). Additionally, we aim to characterize the origin schools of students and the tendencies of young females to choose courses with low female representation, specifically in the fields of exact sciences and engineering. Public educational data from ENEM, SISU, and the School Census are used to feed the data science process, employing preprocessing and exploratory data analysis methodology, as well as observing idiosyncratic variations in gender composition to extract valuable insights from the available data. In this preliminary stage of the work, the ENEM and SISU datasets were combined by creating a unique identifier. This enabled the characterization of the candidates' profiles, higher education institutions, and major courses, noting differences between genders. In the continuation of this work, the exploration of school data will be conducted to apply the idiosyncratic variations method in gender composition.
}
